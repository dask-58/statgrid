\documentclass[conference]{IEEEtran}

% === PACKAGE IMPORTS ===
\usepackage{cite}
\usepackage{graphicx}
\usepackage{amsmath}
\usepackage{booktabs}
\usepackage{url}
\usepackage{multirow}

% --- DOCUMENT METADATA ---
\title{Smart Grid Energy Forecasting: A Statistical and Deep Learning Approach}

\author{\IEEEauthorblockN{Dhruv A Koli, Amritanshu Aditya, D Harsha Vardhan, Kagitha Likhith}
\IEEEauthorblockA{23bc044, 23bc013, 23bc045, 23bc061 \\
Indian Institute of Information Technology, Dharwad}
}

\begin{document}

\maketitle

% === ABSTRACT ===
\begin{abstract}
The smart grid has transformed traditional power systems into data-driven, intelligent networks. This paper reviews statistical and machine learning methods for analyzing and forecasting energy demand. Using the Smart Grid Electricity Marketing Dataset, we perform exploratory analysis and implement a hybrid CNN-BiLSTM model for short-term forecasting. The proposed model achieves a Root Mean Squared Error (RMSE) of 2143.0, outperforming a baseline ARIMA model. The study shows how combining classical statistics with deep learning improves forecasting accuracy and supports better decision-making in modern energy systems.
\end{abstract}

% % === KEYWORDS ===
% \begin{IEEEkeywords}
% Smart Grid, Statistical Analysis, Machine Learning, Deep Learning, Energy Forecasting, Hypothesis Testing, Exploratory Data Analysis, IEEE.
% \end{IEEEkeywords}

\section{Introduction: The Smart Grid as a Data Ecosystem}
The smart grid represents a paradigm shift in energy infrastructure, evolving from a centralized, unidirectional system to a decentralized, intelligent network integrated with advanced communication and data acquisition technologies. This transformation has unlocked an unprecedented flow of high-granularity data from sources like Advanced Metering Infrastructure (AMI). The operational intelligence of the modern grid is an emergent quality derived from the sophisticated application of statistical and computational methodologies to this data deluge. This report synthesizes these techniques, illustrating their application from foundational data analysis to advanced predictive modeling using a real-world dataset.

\IEEEpeerreviewmaketitle

\section{Dataset and Exploratory Data Analysis}
The analyses and models discussed in this paper are grounded in the "Smart Grid Electricity Marketing Dataset" available on Kaggle. This dataset provides a rich, multivariate time-series context for exploring the statistical methods relevant to grid management.

\subsection{Dataset Description}
The dataset contains time-stamped records with variables crucial for energy forecasting. A summary of the key numerical variables is provided in Table \ref{tab:desc_stats}.

\begin{table}[htbp]
\centering
\caption{Descriptive Statistics of Key Numerical Variables}
\label{tab:desc_stats}
\begin{tabular}{lrrrr}
\toprule
 & historical\_avg\_demand & temperature & humidity & price\_signal \\
\midrule
count & 720.00 & 720.00 & 720.00 & 720.00 \\
mean & 0.46 & 0.46 & 0.51 & 0.52 \\
std & 0.19 & 0.14 & 0.29 & 0.20 \\
min & 0.00 & 0.00 & 0.00 & 0.00 \\
25\% & 0.32 & 0.36 & 0.24 & 0.38 \\
50\% & 0.49 & 0.46 & 0.52 & 0.52 \\
75\% & 0.61 & 0.55 & 0.75 & 0.67 \\
max & 1.00 & 1.00 & 1.00 & 1.00 \\
\bottomrule
\end{tabular}
\end{table}

\subsection{Exploratory Analysis Findings}
Initial exploratory data analysis (EDA) is critical for understanding the data's underlying structure. Our analysis reveals distinct consumption patterns, including clear seasonal effects on energy demand (Fig. \ref{fig:demand_time_series}) and significant differences in demand distribution across consumer types (Fig. \ref{fig:demand_by_consumer}). A correlation analysis, summarized in the heatmap in Fig. \ref{fig:corr_heatmap}, indicates a strong positive linear relationship between \texttt{temperature} and \texttt{historical\_avg\_demand}, confirming its importance as a predictive feature.

\begin{figure}[h]
  \centering
  \includegraphics[width=0.9\columnwidth]{figures/demand_over_time.png}
  \caption{Historical average energy demand from the dataset, showing clear daily and weekly seasonality.}
  \label{fig:demand_time_series}
\end{figure}

\begin{figure}[h]
  \centering
  \includegraphics[width=0.9\columnwidth]{figures/demand_boxplot.png}
  \caption{Distribution of energy demand by consumer type. Industrial consumers show a higher median and greater variance.}
  \label{fig:demand_by_consumer}
\end{figure}

\begin{figure}[h]
  \centering
  \includegraphics[width=0.9\columnwidth]{figures/correlation_heatmap.png}
  \caption{Correlation matrix heatmap of numerical features. The strong positive correlation (red) between temperature and demand is notable.}
  \label{fig:corr_heatmap}
\end{figure}

% === NEW SECTION ===
\section{Problem Statement and Proposed Solution}
This section outlines the core forecasting challenge addressed by this paper and the technical methodology proposed to solve it.

\subsection{Problem Statement}
The primary objective is to develop a highly accurate model for short-term electricity demand forecasting using the provided multivariate time-series dataset. Accurate forecasting is critical for energy providers to ensure grid stability, optimize resource allocation, and manage costs effectively. The inherent non-linearity and complex temporal dependencies in energy consumption data present a significant challenge for traditional forecasting methods.

\subsection{Technical Approach and Methodology}
To address this challenge, we propose a deep learning solution and outline our technical stack as follows:
\begin{itemize}
    \item \textbf{Core Algorithm:} A hybrid model combining a Convolutional Neural Network (CNN) for feature extraction from short-term data patterns and a Bidirectional Long Short-Term Memory (BiLSTM) network to capture long-range temporal dependencies.
    \item \textbf{Technology Stack:} The implementation will leverage Python, utilizing \textbf{Pandas} for data manipulation, \textbf{Matplotlib} and \textbf{Seaborn} for visualization, \textbf{Scikit-learn} for data preprocessing, and \textbf{TensorFlow} with the Keras API for building and training the deep learning model.
    \item \textbf{Baseline for Comparison:} To validate the effectiveness of our proposed model, its performance will be benchmarked against a classical statistical forecasting model, specifically an Autoregressive Integrated Moving Average (ARIMA) model.
\end{itemize}

\section{Predictive Methodologies for Grid Operation}
Predictive analytics transforms historical data into forward-looking intelligence. The high volatility and complex patterns in smart meter data often necessitate advanced modeling techniques.

\subsection{Application: Hybrid Deep Learning for Load Forecasting}
This section details a practical roadmap for applying a hybrid deep learning model for short-term energy forecasting, inspired by Al-Shargabi et al. and using our selected dataset.

\subsubsection{Feature Engineering}
Raw data columns were transformed to be more suitable for the deep learning model, as summarized in Table \ref{tab:feature_eng}.

\begin{table}[htbp]
\centering
\caption{Summary of Feature Engineering Techniques}
\label{tab:feature_eng}
\begin{tabular}{@{}ll@{}}
\toprule
\textbf{Original Feature} & \textbf{Engineered Feature(s) / Rationale} \\ \midrule
\texttt{timestamp} & Hour of Day, Day of Week (captures time patterns) \\ \addlinespace
\texttt{consumer\_type} & One-Hot Encoded Columns \\ \bottomrule
\end{tabular}
\end{table}

\subsubsection{Proposed Model Architecture}
The proposed model is a hybrid Convolutional Neural Network (CNN) and Bidirectional Long Short-Term Memory (BiLSTM) network. The CNN layers are designed to extract salient, short-term features from input subsequences, while the BiLSTM layers model the longer-term temporal dependencies that are evident in the data.

\subsubsection{Model Input Preparation}
Time-series forecasting with neural networks requires restructuring the data into a supervised learning format. This is achieved using a "sliding window" approach, where a sequence of past observations (e.g., the last 24 hours of all features) is used as the input ($X$), and the demand at a future time step is the target to predict ($y$). All numerical features are normalized to a [0, 1] range.

\subsubsection{Training and Evaluation Strategy}
The dataset is split chronologically into training, validation, and test sets. Performance is measured using Mean Absolute Error (MAE), Root Mean Squared Error (RMSE), and Mean Absolute Percentage Error (MAPE), and is compared against a simpler baseline model (e.g., ARIMA).

\section{Statistical Evaluation Frameworks}
Evaluating the impact of grid factors requires a formal statistical framework.

\subsection{Parametric Hypothesis Tests}
Parametric tests are powerful but assume the data follows a specific distribution.
\begin{itemize}
    \item \textbf{Independent T-Test:} Compares the means of two independent groups.  
    \item \textbf{ANOVA:} Compares the means of three or more groups by analyzing variance.  
\end{itemize}

\subsection{Non-Parametric Hypothesis Tests}
When parametric assumptions are not met, the \textbf{Wilcoxon Rank-Sum Test} is a robust alternative.

\subsection{Categorical Data Analysis}
The \textbf{Chi-Square Test of Independence} determines if an association exists between two categorical variables.

\subsection{Hypothesis Testing Framework}
To validate key observations from the exploratory data analysis, we formulate and test the following hypotheses:

\subsubsection{Hypothesis A: Impact of Time of Week on Energy Demand}
\textbf{Objective:} To determine if there is a statistically significant difference between the mean energy demand on weekdays and weekends.

\textbf{Null Hypothesis (H₀):} The mean energy demand on weekdays is equal to that on weekends. $\mu_{\text{weekday}} = \mu_{\text{weekend}}$

\textbf{Alternative Hypothesis (Hₐ):} The mean energy demand on weekdays is not equal to that on weekends. $\mu_{\text{weekday}} \neq \mu_{\text{weekend}}$

\textbf{Test Used:} Independent Samples t-test and Mann-Whitney U test (non-parametric alternative)

\subsubsection{Hypothesis B: Impact of Consumer Type on Energy Demand}
\textbf{Objective:} To determine if there are statistically significant differences in mean energy demand across different consumer types (residential, commercial, industrial).

\textbf{Null Hypothesis (H₀):} The mean energy demand is equal across all consumer types. $\mu_{\text{residential}} = \mu_{\text{commercial}} = \mu_{\text{industrial}}$

\textbf{Alternative Hypothesis (Hₐ):} At least one consumer type has a different mean energy demand than the others.

\textbf{Test Used:} One-Way ANOVA and Kruskal-Wallis H test (non-parametric alternative)

\subsubsection{Hypothesis C: Relationship between Temperature and Energy Demand}
\textbf{Objective:} To determine if there is a statistically significant positive correlation between temperature and energy demand.

\textbf{Null Hypothesis (H₀):} There is no correlation between temperature and energy demand. $\rho = 0$

\textbf{Alternative Hypothesis (Hₐ):} There is a positive correlation between temperature and energy demand. $\rho > 0$

\textbf{Test Used:} Pearson correlation coefficient test and Spearman rank correlation test (non-parametric alternative)

\section{Results}
\begin{table}[htbp]
\centering
\caption{Independent T-Test for Weekend vs. Weekday Demand}
\label{tab:ttest_results}
\begin{tabular}{@{}lccc@{}}
\toprule
\textbf{Group} & \textbf{N} & \textbf{Mean Demand} & \textbf{t-statistic} \\ \midrule
Weekday        & 528        & 0.52                 & \multicolumn{1}{c}{\multirow{2}{*}{16.84}} \\
Weekend        & 192        & 0.29                 & \multicolumn{1}{c}{} \\ \bottomrule
\multicolumn{4}{l}{\textit{Note: p-value < 0.001}}
\end{tabular}
\end{table}

\begin{table}[htbp]
\centering
\caption{Performance Metrics for Forecasting Models}
\label{tab:ml_results}
\begin{tabular}{@{}lccc@{}}
\toprule
\textbf{Model} & \textbf{MAE} & \textbf{RMSE} & \textbf{MAPE (\%)} \\ \midrule
Baseline (ARIMA) & 2568.5 & 3366.9 & - \\
CNN-BiLSTM (Proposed) & 1900.2 & 2143.0 & - \\ \bottomrule
\end{tabular}
\end{table}

% ADD IN ARTICLE
% \begin{figure}[htbp]
%   \centering
%   \includegraphics[width=0.9\columnwidth]{figures/prediction_plot.png}
%   \caption{Predicted vs. Actual Energy Consumption on Test Set.}
%   \label{fig:prediction_plot}
% \end{figure}

\section{Discussion}
The results from our hypothesis tests confirm that factors like time of week and consumer type have a statistically significant impact on energy demand. The CNN-BiLSTM model achieved an RMSE of 2143.0, significantly outperforming the ARIMA baseline (RMSE 3366.9). This shows the value of deep learning for capturing complex, non-linear demand dynamics. Future work should explore interpretable AI for critical grid applications.

\section{Conclusion}
The integration of statistical science and machine learning is a core competency for the operation of the modern smart grid. This report has outlined fundamental statistical methodologies and demonstrated their application through an exploratory analysis of a real-world dataset. Furthermore, we have detailed a practical deep learning approach to energy forecasting. A disciplined, evidence-based approach, combining both classical statistics for inference and modern machine learning for prediction, is essential for optimizing grid performance and realizing the full potential of a data-driven energy future.

% === REFERENCES ===
\begin{thebibliography}{99}

\bibitem{ercik2023}
Ü. Erçik and M. Dirik, 
\textit{Data Analysis for Smart Grid and Communication Technologies}, 
April 2023.

\bibitem{gomez2025}
W. Gomez, F.-K. Wang, and S.-H. Sheu, 
``Short-term smart grid energy forecasting using a hybrid deep learning method on univariate and multivariate data sets,'' 
August 2025.

\bibitem{kaur2022}
D. Kaur, S. N. Islam, M. A. Mahmud, M. E. Haque, and Z. Y. Dong, 
``Energy forecasting in smart grid systems: recent advancements in probabilistic deep learning,'' 
July 2022.

\bibitem{fang2012}
X. Fang, S. Misra, G. Xue, and D. Yang, 
``Smart Grid — The New and Improved Power Grid: A Survey,'' 
January 2012.

\bibitem{ieee_article}
IEEE Innovation at Work, 
``Smart Grid: Transforming Renewable Energy,'' 
[Online]. Available: \url{https://innovationatwork.ieee.org/smart-grid-transforming-renewable-energy/}.

\bibitem{malik2023}
P. K. Malik and A. Alkhayyat, 
``Data Analytics for Smart Grids: Applications to Improve Performance, Optimize Energy Consumption, and Gain Insights,'' 
November 2023.

\bibitem{sampath2020}
L. P. M. I. Sampath, Y. Jiawei, and H. B. Gooi, 
``Peer-to-Peer Energy Trading in Smart Grid Considering Power Losses and Network Fees,'' 
\textit{IEEE Transactions on Smart Grid}, May 2020.

\bibitem{paudel2020}
A. Paudel, Y. Jiawei, and H. B. Gooi, 
``Peer-to-Peer Energy Trading in Smart Grids Considering Network Utilization Fees,'' 
August 2020.

\bibitem{azad2019}
S. A. Azad, F. Sabrina, and S. A. Wasimi, 
``Transformation of Smart Grid using Machine Learning,'' 
November 2019.

\bibitem{dong2024}
Q. Dong, R. Huang, C. Cui, D. Towey, L. Zhou, J. Tian, and J. Wang, 
``Short-Term Electricity-Load Forecasting by Deep Learning: A Comprehensive Survey,'' 
August 2024.

\end{thebibliography}
\end{document}
