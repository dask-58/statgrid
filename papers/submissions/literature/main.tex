% =========================================================================
% === IEEEtran LaTeX Class for Journals -- V1.8b
% ===
% === This is a professional, single-file template for your research paper.
% === References are included directly at the end of the document.
% =========================================================================
\documentclass[journal]{IEEEtran}

% --- CORE PACKAGES ---
\usepackage{cite}           % Handles citation compression (e.g., [1]-[3])
\usepackage{graphicx}       % For including graphics
\usepackage[cmex10]{amsmath} % Advanced math environments
\usepackage{amsfonts}       % For math fonts
\usepackage{algorithmic}    % For algorithm formatting
\usepackage{array}          % For extending table and array environments
\usepackage{url}            % For URL formatting

% --- UTILITY PACKAGES ---
\usepackage{booktabs}       % For professional-looking tables (e.g., \toprule, \midrule, \bottomrule)
\usepackage{hyperref}       % For creating hyperlinks in the document (optional, but good practice)
\usepackage{balance}        % To balance the columns on the last page

% Correct bad hyphenation here
\hyphenation{op-tical net-works semi-conduc-tor}


% =========================================================================
% --- DOCUMENT START ---
% =========================================================================
\begin{document}

% --- TITLE ---
% The title is from our project proposal.
\title{A Bayesian Deep Learning Framework for Probabilistic Smart Grid Energy Forecasting}

% --- AUTHOR INFORMATION ---
% Populated with your team's details from the project registration form.
\author{
    \IEEEauthorblockA{
        Department of Computer Science and Engineering,
        Indian Institute of Information Technology (IIIT) Dharwad\\
        Dharwad, Karnataka 580009, India\\
        \{23bcs044, 23bcs013, 23bcs045, 23bcs061\}@iiitdwd.ac.in
    }//
    \IEEEauthorblockN{
        Dhruv Anand Koli,
        Amritanshu Aditya,
        D Harsha Vardhan
        K Likhith
    }
\thanks{Manuscript received August 29, 2025. This work was conducted as part of the Statistics for Computer Science course project, guided by Prof. [Professor's Name].}
}

% --- RUNNING HEADERS ---
% The paper headers
\markboth{Journal of Smart Grid Technology,~Vol.~XX, No.~X, August~2025}%
{Koli \MakeLowercase{\textit{et al.}}: A Bayesian Deep Learning Framework for Probabilistic Forecasting}

% --- MAKE TITLE ---
% This command generates the title based on the information above.
\maketitle


% =========================================================================
% --- ABSTRACT & KEYWORDS ---
% =========================================================================

\begin{abstract}
% The abstract is a concise, self-contained summary of the entire paper.
% Aim for 150-250 words.
% 1. CONTEXT: Briefly introduce the importance of probabilistic energy forecasting.
% 2. PROBLEM: State the limitations of existing point-forecasting models.
% 3. METHODOLOGY: Describe your novel hybrid VAR and Bayesian BiLSTM-AM framework.
% 4. RESULTS: Summarize the key findings (e.g., superior performance in CRPS, PICP).
% 5. CONCLUSION: State the main contribution and its implications for smart grid management.
\end{abstract}

\begin{IEEEkeywords}
% List 4-5 keywords in alphabetical order, separated by commas.
Bayesian deep learning, energy forecasting, probabilistic forecasting, smart grid, time series analysis, uncertainty quantification.
\end{IEEEkeywords}


% =========================================================================
% --- MAIN BODY OF THE PAPER ---
% =========================================================================

% --- I. INTRODUCTION ---
\section{Introduction}
\label{sec:introduction}
\IEEEPARstart{H}{ere}, you will set the stage for your research.
%
% Structure:
% 1. Start with the broad context: The global shift towards renewable energy and the challenges it creates for grid stability.
% 2. Narrow down to the specific problem: The need for accurate and reliable energy demand forecasting, emphasizing the inadequacy of single-point predictions in handling the volatility of renewables.
% 3. State the motivation: Introduce the concept of probabilistic forecasting as a superior alternative for risk management and decision-making.
% 4. Clearly state your contribution: "In this paper, we propose a novel hybrid statistical-deep learning framework that integrates..."
% 5. Outline the paper: Briefly describe the structure of the remaining sections. "Section II reviews related work... Section III details our methodology..."
%
% Remember to cite relevant work using \cite{key_paper_tag}. For example, you can cite the base paper as \cite{Gomez2025}.


% --- II. RELATED WORK ---
\section{Related Work}
\label{sec:related_work}
%
% This section positions your work within the existing literature.
%
% Structure:
% 1. Start with traditional time-series models (ARIMA, VAR) for energy forecasting and discuss their limitations.
% 2. Discuss the rise of machine learning and deep learning models (LSTMs, Transformers), referencing the base paper \cite{Gomez2025} and others like \cite{Lim2021}.
% 3. Focus on the literature for uncertainty quantification in forecasting, comparing methods like MC Dropout \cite{Gal2016} with more formal Bayesian approaches.
% 4. Conclude by identifying the research gap: "While hybrid models have shown promise, few have integrated classical statistical models with fully Bayesian deep learning architectures for a principled probabilistic forecast." This clearly justifies your novel approach.


% --- III. PROPOSED METHODOLOGY ---
\section{Proposed Methodology}
\label{sec:methodology}
%
% This is the core technical section. Describe your framework in detail.
% Use sub-sections to keep it organized. A figure showing the model architecture is highly recommended here.
%
\subsection{Data Preprocessing and Decomposition}
%
% 1. Describe the dataset you are using.
% 2. Explain the CEEMDAN decomposition process, referencing the base paper \cite{Gomez2025}.
% 3. Mention the use of their "Algorithm 1" for finding the optimal number of IMFs.
%
\subsection{Hybrid Probabilistic Model Architecture}
%
% 1. Explain the two-stage hybrid model for each IMF.
% 2. Describe the Vector Autoregression (VAR) component for capturing linear dependencies. Provide the relevant equations.
% 3. Detail the Bayesian BiLSTM-AM component for modeling the non-linear residuals. Explain that you learn a distribution over the weights, not point estimates.
%
\subsection{Model Training and Inference}
%
% 1. Explain the training process. Mention using Variational Inference (VI) to approximate the posterior distribution of the network weights.
% 2. Describe how the final predictive distribution is synthesized by combining the VAR output and the Bayesian BiLSTM's probabilistic output.


% --- IV. EXPERIMENTAL SETUP ---
\section{Experimental Setup}
\label{sec:experiments}
%
% This section details how you evaluated your model, ensuring your results are reproducible.
%
\subsection{Dataset}
%
% Provide a detailed description of the Smart Grid Electricity Marketing Dataset. Include its size, features, time granularity, and the train/validation/test split.
%
\subsection{Baseline Models}
%
% List the models you are comparing against. This must include the model from the base paper.
% - ARIMA/VAR
% - Standard BiLSTM-AM-DNN (from Gomez et al. \cite{Gomez2025})
% - ... any other models you choose to compare with.
%
\subsection{Evaluation Metrics}
%
% Define the metrics used for evaluation. Separate them into point and probabilistic metrics.
% - Point Metrics: MAE, RMSE. Provide equations.
% - Probabilistic Metrics: CRPS, PICP, Winkler Score. Provide equations and explain what they measure.
% - Statistical Test: Mention the Giacomini-White (GW) test for conditional predictive ability.


% --- V. RESULTS AND DISCUSSION ---
\section{Results and Discussion}
\label{sec:results}
%
% Present and interpret your findings. Use tables and figures extensively.
\subsection{Point Forecast Accuracy}
%
% Discuss the results from Table \ref{tab:results} in terms of RMSE and MAE. Compare your model's point-forecast performance to the baselines.
%
\subsection{Probabilistic Forecast Quality}
%
% This is a key part of your contribution.
% 1. Analyze the CRPS and PICP scores from Table \ref{tab:results}. Explain why your model performs better.
% 2. Include a figure that visualizes the probabilistic forecasts (a "fan chart" showing prediction intervals) for a segment of the test set, comparing your model's output to the true values and a baseline.
%
\subsection{Ablation Study}
%
% Analyze the contribution of each component of your model.
% - What happens if you remove the VAR component?
% - What happens if you remove the Bayesian layers?
% This demonstrates a deep understanding of your model's architecture.


% --- VI. CONCLUSION ---
\section{Conclusion}
\label{sec:conclusion}
%
% Summarize your work and its impact.
% 1. Briefly reiterate the problem and your solution.
% 2. Summarize the key findings and the main contribution of your paper.
% 3. Discuss the practical implications of your work for smart grid operators.
% 4. Suggest promising directions for future work (e.g., applying the framework to other domains, exploring different Bayesian inference techniques).



% =========================================================================
% --- REFERENCES ---
% =========================================================================
% The references are now included directly in this file.
\begin{thebibliography}{1}
\balance

\bibitem{Gomez2025}
W. Gomez, F.-K. Wang, and S.-H. Sheu, "Short-term smart grid energy forecasting using a hybrid deep learning method on univariate and multivariate data sets," \textit{Energy}, vol. 335, p. 138081, 2025.

\bibitem{Gal2016}
Y. Gal and Z. Ghahramani, "Dropout as a Bayesian approximation: Representing model uncertainty in deep learning," in \textit{Proc. Int. Conf. Machine Learning}, 2016, pp. 1050--1059.

\bibitem{Salinas2020}
D. Salinas, V. Flunkert, J. Gasthaus, and T. Januschowski, "DeepAR: Probabilistic forecasting with autoregressive recurrent networks," \textit{Int. J. Forecasting}, vol. 36, no. 3, pp. 1181--1191, 2020.

\bibitem{Lim2021}
B. Lim, S. Ö. Arık, N. Loeff, and T. Pfister, "Temporal fusion transformers for interpretable multi-horizon time series forecasting," \textit{Int. J. Forecasting}, vol. 37, no. 4, pp. 1748--1764, 2021.

% --- Add more references here in the same format ---
% \bibitem{key}
% Author(s), "Title," \textit{Journal}, vol. X, no. Y, pp. Z-ZZ, Year.

\end{thebibliography}

\end{document}
